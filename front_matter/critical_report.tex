\documentclass{ees}

\begin{document}

\eesTitlePage

\eesCriticalReport{}

\eesToc{
  – &   – & – & Although \B1 is a copy by Joseph Estlinger and therefore may be regarded as authentic, articulation and dynamics are quite inconsistent in this source. In this edition, these differences are only carefully harmonized. \\
  – &   – & –   & –     & The Viennese abridged version of this work (e.g., A-Wn HK.1813, RISM 600101827) contains large cuts, which comprise the following bars:
        \textit{Requiem} (bars 1–12 and 38–45, yielding 27 bars in the short version vs 47 bars in \B1),
        \textit{Te decet} (52–60, 77f, 106f, 113; 54 vs 68),
        \textit{Kyrie} (6–8 (replaced by one new bar), 27–38; 43 vs 29),
        \textit{Dies iræ} (62–65; 66 vs 62),
        \textit{Liber scriptus} (74–81, 146–155, 176–183 (1 new), 200–207 (1 new); 254 vs 222),
        \textit{Domine Jesu Christe} (56–60 (1 new), 83–87 (1 new), 90–92 (2 new); 94 vs 85),
        \textit{Quam olim Abrahæ 1} (replaced by the shorter repeat; 136 vs 41),
        \textit{Hostias} (239–249; 262–266, 273–284 (1 new); 65 vs 38),
        % \textit{Sanctus} (21 vs 25),
        % \textit{Osanna} (11 vs 23),
        % \textit{Benedictus} (35 vs 102),
        % \textit{Osanna 2} (11 vs 13),
        % and \textit{Requiem 2} (27 vs 47).
        There are no cuts in
        \textit{Lacrymosa} (30 bars),
        \textit{Quam olim Abrahæ 2} (41 bars), and
        % \textit{Agnus Dei} (43 bars).
  1 & 12  & vl 2 & 1st \quarterNote\ in \B1: c′8.–c″16 \\
    & 47  & vl 2 & 1st \quarterNote\ in \B1: c′8.–g16 \\
  2 & 36  & vla  & 2nd/3rd \eighthNote\ in \B1: e′8–f′8 \\
    & 40  & vla  & 1st \quarterNote\ in \B1: a′8–c″8 \\
    & 81  & vl 2 & 1st \quarterNote\ in \B1: b′16–g′16–e′16–a′16 \\
    & 195 & vl 2 & 3rd \sixteenthNote\ in \B1: b′16 \\
    & 345 & A    & 2nd \quarterNote\ in \B1: d′4 \\
  3 & 156 & vla  & 4th \quarterNote\ in \B1: d′4 \\
    & 248 & A    & 1st \quarterNote\ in \B1: a′8.–b′16–c″16 \\
}

\eesScore

\end{document}
